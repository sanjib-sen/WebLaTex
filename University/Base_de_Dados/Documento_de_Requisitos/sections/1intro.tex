\section{Introdução}
\label{sec:introducao}
Uma das principais problemáticas encaradas dentro do contexto sociopolítico urbano, principalmente dentro de megalópoles, é a gritante desigualdade encontrada por entre as entranhas de suas ruas, ocasionando em uma quantidade gritante de cidadãos sem acesso ao básico para sua sobrevivência. A origem desta situação é, por muitas vezes, definida erroneamente como vinda da dependência química, entretanto os motivos são diversos mas, fato é, que em regiões específicas de grandes centros urbanos existe uma concentração enorme de pessoas sem moradia fixa que fazem uso de algum entorpecente com frequência. Este quadro força que a abordagem de acolhimento em abrigos e em assistências comunitárias tenham maior sensibilidade quanto a isso, necessitando a observação dos dados de evolução do auxiliado, dos recursos financeiros, dos voluntários e das campanhas de auxílio.

O seguinte projeto se trata de uma iniciativa dos moradores do centro de uma grande metrópole para reestruturação da região da forma mais humanizada possível, essencialmente teremos um \textbf{auxiliado}, o qual será recebido por este coletivo, classificado por auto-declaração e atendido da forma mais personalizada possível.

