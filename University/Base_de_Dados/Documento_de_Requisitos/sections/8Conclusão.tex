\section{Conclusão e considerações finais}

Após um semestre inteiro aprendendo os fundamentos de base de dados podemos concluir, em grupo, alguns aspectos relevantes da disciplina e do trabalho no geral. O tema de ações comunitárias fez o grupo a pensar em situações nas quais a importância dos conhecimentos aprendidos na matéria se destacaria com maior relevância. 

Assim, escolhendo o tema de Auxílios a Dependentes Químicos e Moradores de rua nos fez pensar não somente na disciplina, mas sim como esse projeto poderia servir como base para uma implementação de um Centro Comunitário a fim de auxiliar aqueles que poderiam necessitar de ajuda.

O projeto começou com um dos maiores desafios que foi a criação da ideia e o desenvolvimento do Modelo Entidade Relacionamento (MER). O qual foi o mais complicado para o grupo, mas com os auxílios da Docente e Monitor PAE se mostrou como um dos mais importantes na definição de uma Base de dados. 

No demais, o projeto se mostrou como uma grande forma de fazer o grupo a aprender realizando uma integração com toda a disciplina e também como deve ser implementada uma base dados corretamente, tanto para a academia quanto para o mercado. Dividir o projeto em três partes e cada parte ser sincronizada com o assunto da disciplina se mostrou muito eficaz no aprendizado.

Por fim, o projeto da disciplina é muito rico em informações e conteúdo para fazer o aluno da disciplina aprender muito bem os assuntos de Base de dados.

