\section{Mudanças realizadas em relação a primeira entrega}
\label{sec:entidades}

\begin{enumerate}
    \item Agregações: A modelagem apresentada anteriormente tinha alguns problemas de apresentação. Boa parte das ligações referente às agregações não possuíam as conexões feitas de maneira correta - muitas vezes, a linha não se juntava ao losango interior. Assim , atualizamos o arquivo permitindo o melhor entendimento dessas entidades.
    \item Auxiliado (Dependentes químicos ou moradores de rua): Foi realizada a correção da cardinalidade da entidade morador de rua com a relação de alojamento, uma vez que é mais semântico permitir que uma ONG aloje diversos moradores de rua.
    \item Centro Comunitário: Foi corrigida a modelagem de presidente enquanto atributo. Neste caso, este evento foi recriado como uma relação entre centro comunitário e funcionário. Neste ponto, é importante destacar a criação de um ciclo. Aqui, faz-se indispensável que o presidente presida apenas um centro comunitário do qual faz parte
    \item Doação: A definição de doação torna-se mais clara caso seja modelada como múltiplos itens - como produtos ou monetária. Anteriormente, havia-se criado uma entidade fraca "Patrocínios". Contudo, essa nova abordagem atua de forma mais limpa e direta com os objetivos do trabalho. Com isso, modificamos também sua chave, que agora, ao invés de precisar depender das chaves de suas partes, poderá ser apenas a sua nota fiscal, a qual permite a presença tanto de valores monetários quanto de produtos.
    \item Frentes de ação (Reabilitação, Reestruturação Profissional): A chave fraca de frente foi modelada como o seu "nome". Isso se deve ao analisar o próprio meio empresarial, onde diversas empresas possuem nomes diferentes para frentes de mesma atuação. Por exemplo, a gestão de recursos humanos em algumas é chamada de RH, em outras, "People". Contudo, elas são basicamente do mesmo tipo. Poderia-se tentar modelar sendo o próprio tipo como chave e atributo de especialização, mas isso geraria impactos negativos na modelagem relacional, tendo essa solução apresentada como uma boa maneira de contornar-se o problema. Foi pensado também a definição da chave parcial de frente como a sua assinatura eletrônica, porém, isso a faria ter um identificador único. Semanticamente, é mais correto deixar a frente como dependente do centro, por isso opta-se pela modelagem como entidade fraca.
    \item Terceiros: A utilização de ID's não é recomendada para o MER. Neste caso, foi sugerida uma abordagem em que tanto CPF quanto CNPJ poderiam atuar como chave de Terceiros. Para isso, o valor informado seria tratado na aplicação, através de um RegEx - expressão regular. Nesta, haveriam regras de entrada para determinar o tipo da chave. Além disso, deveria também se ter uma analise à fim de garantir que o uso de CPF seja exclusivo de pessoas físicas, enquanto o uso de CNPJ seria apenas para empresas parceiras e ONG's.
    \item Eventos: Anteriormente, a modelagem de um evento tendo apenas a chave como seu nome impedia que diversos centros possuíssem eventos com o mesmo nome ou que um mesmo centro comunitário refizesse esse evento em outras edições. Agora, sua chave possuí nome, data e local. Com isso, evita-se conflitos que poderiam surgir anteriormente e garante maior liberdade ao centro.

    \item Descrição de entidades e relacionamentos: Foram expandidas e melhor explicadas as entidades e relacionamentos presentes no MER visando a coesão, a coerência e o maior detalhamento, a fim de facilitar a leitura e entendimento do trabalho.
\end{enumerate}